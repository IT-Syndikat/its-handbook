% TODO check for copy-paste errors
% TODO (?) improve typography

% Adjust the spacing values in the following definitions to fit the content on as few pages as possible.

\newcommand{\statutensection}[1]{%
    \vspace{-2em}%
    \section*{\small \S{} #1}%
    \vspace{-1em}%
}

\SetEnumitemKey{statutenenum}{
    topsep=-0.8em,
    partopsep=0.2em,
    itemsep=-0.32em,
    parsep=0.6em,
    leftmargin=2em,
    labelsep=0.3em,
}

\chapter{Statuten}
\label{cha:statuten}

This chapter of the handbook is not translated in order to prevent mistranslations.

\begin{otherlanguage}{german}

\fontfamily{ptm}\selectfont\tiny

\statutensection{1 Name, Size und Tätigkeitsbereich}

\begin{enumerate}[statutenenum]
    \item Der Verein führt den Namen: \textit{Verein zur Förderung des freien Zugangs zu technischer Fort- und Weiterbildung jeglicher Art}
    \item In der Alltagskommunikation kann die Bezeichnung
      ``IT-Syndikat'' verwendet werden.

    \item Er hat seinen Sitz in Innsbruck und erstreckt seine Tätigkeit auf den Planeten Erde, den Cyberspace und das beobachtbare Universum.

    \item Die Errichtung von Zweigvereinen ist nicht beabsichtigt.
\end{enumerate}

\statutensection{2 Zweck}

Der Verein, dessen Taetigkeit nicht auf Gewinn gerichtet ist, bezweckt die Errichtung und Erhaltung von Infrastruktur zur Vermittlung, Erarbeitung und Verbreitung technischen Wissens und handwerklicher Faehigkeiten mit Fokus und Schwerpunkt auf den sozialen und politischen Implikationen dieser Inhalte sowie der zur Durchfuehrung solcher Projekte notwendigen Ablaeufe und Vorgehensstrategien.

\textbf{Folgende Handlungs- und Ethischen Richtlinien bilden die Grundlage der Aktionen des Vereins:}

\begin{itemize}[statutenenum]
    \item Mülle nicht in den Daten anderer Leute.

    \item Öffentliche Daten nützen, private Daten schützen.
        Alle Informationen müssen frei sein.

    \item Der Zugang zu Technik und allem, was einem zeigen kann, wie diese Welt funktioniert, sollte unbegrenzt und vollständig sein.

    \item Misstraue Autoritäten -- fördere Dezentralisierung

    \item Beurteile eine Haeckse oder einen Hacker nach dem, was er oder sie tut und nicht nach üblichen Kriterien wie Aussehen, Alter, ethnischer Zugehörigkeit, Geschlecht oder gesellschaftlicher Stellung.

    \item Man kann mit Technik Kunst und Schönheit schaffen.

    \item Technik kann dein Leben zum Besseren verändern.
\end{itemize}

\statutensection{3 Mittel zur Erreichung des Vereinszwecks}

\begin{enumerate}[statutenenum]
    \item Der Vereinszweck soll durch die in den Abs.\ 2 und 3 angeführten ideellen und materiellen Mittel erreicht werden.

    \item Als ideelle Mittel dienen
        \begin{enumerate}[statutenenum]
            \item Vorträge und (Diskussions-)Veranstaltungen Versammlungen gesellige Zusammenkünfte Kulturpflege Informationationsveranstaltungen
        \end{enumerate}

    \item Die erforderlichen materiellen Mittel sollen aufgebracht werden durch
        \begin{enumerate}[statutenenum]
            \item Beitrittsgebühren und Mitgliedsbeiträge

            \item Erträgnisse aus Veranstaltungen und vereinseigenen Unternehmungen

            \item Spenden, Sammlungen, Vermächtnisse und sonstige Zuwendungen
        \end{enumerate}
\end{enumerate}

\statutensection{4 Arten der Mitgliedschaft}

\begin{enumerate}[statutenenum]
    \item Die Mitglieder des Vereins gliedern sich in ordentliche, außerordentliche und fördernde Mitglieder.

    \item Ordentliche Mitglieder sind jene, die sich voll an der Vereinsarbeit beteiligen.
        Außerordentliche Mitglieder sind solche, die die Vereinstätigkeit vor allem durch Zahlung eines erhöhten Mitgliedsbeitrags fördern.
        Fördernde Mitglieder sind jene, die die Vereinstätigkeit durch Geld- oder Sachspenden fördern.
\end{enumerate}

\statutensection{5 Erwerb der Mitgliedschaft}

\begin{enumerate}[statutenenum]
    \item Mitglieder des Vereins können alle physischen Personen werden.

    \item Über die Aufnahme von ordentlichen, außerordentlichen und fördernden Mitgliedern entscheidet die Vollversammlung.
        Die Aufnahme kann ohne Angabe von Gründen verweigert werden.

    \item Bis zur Entstehung des Vereins erfolgt die vorläufige Aufnahme von ordentlichen und außerordentlichen Mitgliedern durch die Vereinsgründer/Vereinsgründerinnen, im Fall eines bereits bestellten Vorstands durch diesen.
        Diese Mitgliedschaft wird erst mit Entstehung des Vereins wirksam.
        Wird ein Vorstand erst nach Entstehung des Vereins bestellt, erfolgt auch die (definitive) Aufnahme ordentlicher und außerordentlicher Mitglieder bis dahin durch die Gründer/Gründerinnen des Vereins.
\end{enumerate}

\statutensection{6 Beendigung der Mitgliedschaft}

\begin{enumerate}[statutenenum]
    \item Die Mitgliedschaft erlischt durch Tod, durch freiwilligen Austritt und durch Ausschluss.

    \item Der Austritt kann jederzeit erfolgen.
        Er muss der Vollversammlung einen Monat vorher mitgeteilt werden.

    \item Die Vollversammlung kann ein Mitglied ausschließen, wenn dieses trotz zweimaliger schriftlicher Mahnung unter Setzung einer angemessenen Nachfrist länger als drei Monate mit der Zahlung der Mitgliedsbeiträge im Rückstand ist.
        Die Verpflichtung zur Zahlung der fällig gewordenen Mitgliedsbeiträge bleibt hiervon unberührt.
\end{enumerate}

\statutensection{7 Rechte und Pflichten der Mitglieder}

\begin{enumerate}[statutenenum]
    \item Die Mitglieder sind berechtigt, an allen Veranstaltungen des Vereins teilzunehmen und die Einrichtungen des Vereins zu beanspruchen.
        Das Stimmrecht in der Voll- und Generalversammlung sowie das aktive und passive Wahlrecht steht allen ordentlichen Mitgliedern zu.

    \item Jedes Mitglied ist berechtigt, von der Vollversammlung die Ausfolgung der Statuten zu verlangen.

    \item Mindestens ein Zehntel der ordentlichen Mitglieder kann von der Vollversammlung die Einberufung einer Generalversammlung verlangen.

    \item Die Mitglieder sind in jeder Generalversammlung vom Vorstand über die Tätigkeit und finanzielle Gebarung des Vereins zu informieren.
        Wenn mindestens ein Zehntel der ordentlichen Mitglieder dies unter Angabe von Gründen verlangt, hat der Vorstand den betreffenden Mitgliedern eine solche Information auch sonst binnen vier Wochen zu geben.

    \item Die Mitglieder sind vom Vorstand über den geprüften Rechnungsabschluss (Rechnungslegung) zu informieren.
        Geschieht dies in der Generalversammlung, sind die Rechnungsprüfer/Rechnungsprüferinnen einzubinden.

    \item Die Mitglieder sind verpflichtet, die Interessen des Vereins nach Kräften zu fördern und alles zu unterlassen, wodurch das Ansehen und der Zweck des Vereins Abbruch erleiden könnte.
        Sie haben die Vereinsstatuten und die Beschlüsse der Vereinsorgane zu beachten.
        Die ordentlichen und außerordentlichen Mitglieder sind zur pünktlichen Zahlung der Beitrittsgebühr und der Mitgliedsbeiträge in der von der Generalversammlung beschlossenen Höhe verpflichtet.
\end{enumerate}

\statutensection{8 Vereinsorgane}

Organe des Vereins sind die Vollversammlung (\S\ 99 und \S\ 10), die Generalversammlung (\S\ 11 und \S\ 12), der Vorstand (\S\S\ 13 bis 15), die Rechnungsprüfer (\S\ 16) und das Schiedsgericht (\S\ 17).

\statutensection{9 Vollversammlung}

\begin{enumerate}[statutenenum]
    \item Eine Vollversammlung findet regelmäßig statt.

    \item Eine Vollversammlung findet auf
        \begin{enumerate}[statutenenum]
            \item Anforderung eines ordentlichen Mitgliedes

            \item Beschluss einer Vollversammlung statt.
        \end{enumerate}

    \item Zu den Vollversammlungen sind alle Mitglieder mindestens zwei Tage vor dem Termin schriftlich, per Aushang im Vereinslokal und per E-Mail (an die vom Mitglied dem Verein bekanntgegebene E-Mail-Adresse) einzuladen.
        Die Einberufung erfolgt durch den Vorstand, durch die/eine/n Rechnungsprüfer/Rechnungsprüferin oder ein ordentliches Mitglied.

    \item Die Tagespunkte einer Vollversammlung können auch noch im Laufe der Vollversammlung eingebracht werden.

    \item Bei der Vollversammlung sind alle Mitglieder teilnahmeberechtigt.
        Stimmberechtigt sind alle ordentliche Mitglieder.
        Jedes Mitglied hat eine Stimme.
        Die Übertragung des Stimmrechts auf ein anderes Mitglied im Wege einer schriftlichen Bevollmächtigung ist zulässig.

    \item Die Vollversammlung ist nach dem festgesetzten Termin mit erscheinen von mindestens 1/10tel der ordentlichen Mitglieder beschlussfähig.

    \item Den Vorsitz in der Vollversammlung führt der/die Obmann/Obfrau, in dessen/deren Verhinderung sein/e/ihr/e Stellvertreter/in.
        Wenn auch diese/r verhindert ist, so führt das an Jahren älteste anwesende Vereinsmitglied oder ein beliebiges anders ordentliches Mitglied den Vorsitz.

    \item Die Beschlussfassungen in der Vollversammlung benötigen die Zustimmung aller abgegebenen gültigen Stimmen.
\end{enumerate}

\statutensection{10 Aufgaben der Vollversammlung}

Der Vollversammlung sind folgende Aufgaben vorbehalten:

\begin{enumerate}[statutenenum]
    \item alle den Verein betreffenden Entscheidungen
        \begin{enumerate}[statutenenum]
            \item Eine Entscheidung der Vollversammlung ist nötig, wenn Projekte im Namen des Vereins durchgeführt oder vom Verein subventioniert werden.
        \end{enumerate}
    \item Aufnahme und Ausschluss von ordentlichen, außerordentlichen und fördernden Mitgliedern.
\end{enumerate}

\statutensection{11 Generalversammlung}

\begin{enumerate}[statutenenum]
    \item Die Generalversammlung ist die „Mitgliederversammlung“ im Sinne des Vereinsgesetzes 2002.
        Eine ordentliche Generalversammlung findet jedes Jahr statt.

    \item Eine außerordentliche Generalversammlung findet auf
        \begin{enumerate}[statutenenum]
            \item Beschluss der Vollversammlung oder der ordentlichen Generalversammlung,

            \item schriftlichen Antrag von mindestens einem Zehntel der Mitglieder,

            \item Verlangen der Rechnungsprüfer/innen (\S\ 21 Abs.\ 5 erster Satz VereinsG),

            \item Beschluss der/eines Rechnungsprüfer/s oder der/einer Rechnungsprüferinnen/Rechnungsprüferin (\S\ 21 Abs.\ 5 zweiter Satz VereinsG, \S\ 13 Abs.\ 2 dritter Satz dieser Statuten),

            \item Beschluss eines/einer gerichtlich bestellten Kurators/Kuratorin (\S\ 13 Abs.\ 2 letzter Satz dieser Statuten)
        \end{enumerate}
        binnen vier Wochen statt.

    \item Sowohl zu den ordentlichen wie auch zu den außerordentlichen Generalversammlungen sind alle Mitglieder mindestens zwei Wochen vor dem Termin schriftlich, mittels Telefax oder per E-Mail (an die vom Mitglied dem Verein bekanntgegebene Fax-Nummer oder E-Mail-Adresse) einzuladen.
        Die Anberaumung der Generalversammlung hat unter Angabe der Tagesordnung zu erfolgen.
        Die Einberufung erfolgt durch den Vorstand (Abs.\ 1 und Abs.\ 2 lit.\ a--c), durch die/einen Rechnungsprüfer (Abs.\ 2 lit.\ d) oder durch einen gerichtlich bestellten Kurator (Abs.\ 2 lit.\ e).

    \item Anträge zur Generalversammlung sind mindestens drei Tage vor dem Termin der Generalversammlung beim Vorstand schriftlich, mittels Telefax oder per EMail einzureichen.

    \item Gültige Beschlüsse -- ausgenommen solche über einen Antrag auf Einberufung einer außerordentlichen Generalversammlung -- können nur zur Tagesordnung gefasst werden.

    \item Bei der Generalversammlung sind alle Mitglieder teilnahmeberechtigt.
        Stimmberechtigt sind alle Mitglieder.
        Jedes Mitglied hat eine Stimme.
        Die Übertragung des Stimmrechts auf ein anderes Mitglied im Wege einer schriftlichen Bevollmächtigung ist zulässig.

    \item Die Generalversammlung ist ohne Rücksicht auf die Anzahl der Erschienenen beschlussfähig.

    \item Die Wahlen und die Beschlussfassungen in der Generalversammlung erfolgen in der Regel mit einfacher Mehrheit der abgegebenen gültigen Stimmen.
        Beschlüsse, mit denen das Statut des Vereins geändert oder der Verein aufgelöst werden soll, bedürfen jedoch einer qualifizierten Mehrheit von zwei Dritteln der abgegebenen gültigen Stimmen.

    \item Den Vorsitz in der Generalversammlung führt der/die Obmann/Obfrau, in dessen/deren Verhinderung sein/e/ihr/e Stellvertreter/in.
        Wenn auch diese/r verhindert ist, so führt das an Jahren älteste anwesende Vorstandsmitglied den Vorsitz.
\end{enumerate}

\statutensection{12 Aufgaben der Generalversammlung}

Der Generalversammlung sind folgende Aufgaben vorbehalten:

\begin{enumerate}[statutenenum]
    \item Beschlussfassung über den Voranschlag.

    \item Entgegennahme und Genehmigung des Rechenschaftsberichts und des Rechnungsabschlusses unter Einbindung der Rechnungsprüfer.

    \item Wahl und Enthebung der Mitglieder des Vorstands und der Rechnungsprüfer.

    \item Genehmigung von Rechtsgeschäften zwischen Rechnungsprüfern und Verein.

    \item Entlastung des Vorstands.

    \item Festsetzung der Höhe der Beitrittsgebühr und der Mitgliedsbeiträge für ordentliche und für außerordentliche Mitglieder.

    \item Verleihung und Aberkennung der Ehrenmitgliedschaft.

    \item Beschlussfassung über Statutenänderungen und die freiwillige Auflösung des Vereins.

    \item Beratung und Beschlussfassung über sonstige auf der Tagesordnung stehende Fragen.
\end{enumerate}

\statutensection{13 Vorstand}

\begin{enumerate}[statutenenum]
    \item Der Vorstand besteht aus sechs Mitgliedern, und zwar aus Obmann/Obfrau und Stellvertreter/in, Schriftführer/in und Stellvertreter/in sowie Kassier/in und Stellvertreter/in.

    \item Der Vorstand wird von der Generalversammlung gewählt.
        Die Vollversammlung hat bei Ausscheiden eines gewählten Mitglieds das Recht, an seine Stelle ein anderes wählbares Mitglied zu kooptieren, wozu die nachträgliche Genehmigung in der nächstfolgenden Generalversammlung einzuholen ist.
        Fällt der Vorstand ohne Ergänzung durch Kooptierung überhaupt oder auf unvorhersehbar lange Zeit aus, so ist jeder/jede Rechnungsprüfer/in verpflichtet, unverzüglich eine außerordentliche Generalversammlung zum Zweck der Neuwahl eines Vorstands einzuberufen.
        Sollten auch die Rechnungsprüfer/innen handlungsunfähig sein, hat jedes ordentliche Mitglied, das die Notsituation erkennt, unverzüglich die Bestellung eines/einer Kurators/Kuratorin beim zuständigen Gericht zu beantragen, der umgehend eine außerordentliche Generalversammlung einzuberufen hat.

    \item Die Funktionsperiode des Vorstands beträgt ein Jahr.
        Wiederwahl ist möglich.
        Jede Funktion im Vorstand ist persönlich auszuüben.

    \item Der Vorstand wird vom Obmann/von der Obfrau, bei Verhinderung von seinem/seiner/ihrem/ihrer Stellvertreter/in, schriftlich oder mündlich einberufen.
        Ist auch diese/r auf unvorhersehbar lange Zeit verhindert, darf jedes sonstige Vorstandsmitglied den Vorstand einberufen.

    \item Der Vorstand ist beschlussfähig, wenn alle seine Mitglieder eingeladen wurden und mindestens die Hälfte von ihnen anwesend ist.

    \item Der Vorstand fasst seine Beschlüsse mit einfacher Stimmenmehrheit;
        bei Stimmengleichheit gibt die Stimme des/der Vorsitzenden den Ausschlag.

    \item Den Vorsitz führt der/die Obmann/Obfrau, bei Verhinderung sein/e/ihr/e Stellvertreter/in.
        Ist auch diese/r verhindert, obliegt der Vorsitz dem an Jahren ältesten anwesenden Vorstandsmitglied oder jenem Vorstandsmitglied, das die übrigen Vorstandsmitglieder mehrheitlich dazu bestimmen.

    \item Außer durch den Tod und Ablauf der Funktionsperiode (Abs.\ 3) erlischt die Funktion eines Vorstandsmitglieds durch Enthebung (Abs.\ 9) und Rücktritt (Abs.\ 10).

    \item Die Generalversammlung kann jederzeit den gesamten Vorstand oder einzelne seiner Mitglieder entheben.
        Die Enthebung tritt mit Bestellung des neuen Vorstands bzw Vorstandsmitglieds in Kraft.

    \item Die Vorstandsmitglieder können jederzeit schriftlich ihren Rücktritt erklären.
        Die Rücktrittserklärung ist an den Vorstand, im Falle des Rücktritts des gesamten Vorstands an die Generalversammlung zu richten.
        Der Rücktritt wird erst mit Wahl bzw Kooptierung (Abs.\ 2) eines Nachfolgers wirksam.
\end{enumerate}

\statutensection{14 Aufgaben des Vorstands}

Dem Vorstand obliegt die Vertretung der Vereins gegenüber der Öffentlichkeit und die Exekution der, durch die Vollversammlung getroffenen, Entschlüsse.
Ihm kommen alle Aufgaben zu, die nicht durch die Statuten einem anderen Vereinsorgan zugewiesen sind.
In seinen Wirkungsbereich fallen insbesondere folgende Angelegenheiten:

\begin{enumerate}[statutenenum]
    \item Einrichtung eines den Anforderungen des Vereins entsprechenden Rechnungswesens mit laufender Aufzeichnung der Einnahmen/Ausgaben und Führung eines Vermögensverzeichnisses als Mindesterfordernis.

    \item Erstellung des Jahresvoranschlags, des Rechenschaftsberichts und des Rechnungsabschlusses.

    \item Vorbereitung und Einberufung der Generalversammlung in den Fällen des \S\ 9 Abs.\ 1 und Abs.\ 2 lit.\ a--c dieser Statuten.

    \item Information der Vereinsmitglieder über die Vereinstätigkeit, die Vereinsgebarung und den geprüften Rechnungsabschluss.

    \item Verwaltung des Vereinsvermögens.

    \item Aufnahme und Kündigung von Angestellten des Vereins.
\end{enumerate}

\statutensection{15 Besondere Obliegenheiten einzelner Vorstandsmitglieder}

\begin{enumerate}[statutenenum]
    \item Der/die Obmann/Obfrau führt die laufenden Geschäfte des Vereins.
        Der/die Schriftführer/in unterstützt den/die Obmann/Obfrau bei der Führung der Vereinsgeschäfte.

    \item Der/die Obmann/Obfrau vertritt den Verein nach außen.
        Der/die Schriftführer/in unterstützt den/die Obmann/Obfrau dabei.
        Schriftliche Ausfertigungen des Vereins bedürfen zu ihrer Gültigkeit der Unterschriften des/der Obmanns/Obfrau und des Schriftführers/der Schriftführerin, in Geldangelegenheiten (vermögenswerte Dispositionen) des/der Obmanns/Obfrau und des Kassiers/der Kassierin.
        Rechtsgeschäfte zwischen Vorstandsmitgliedern und Verein bedürfen der Zustimmung der Vollversammlung.

    \item Rechtsgeschäftliche Bevollmächtigungen, den Verein nach außen zu vertreten bzw.\ für ihn zu zeichnen, können ausschließlich von den in Abs.\ 2 genannten Vorstandsmitgliedern erteilt werden.

    \item Bei Gefahr im Verzug ist der/die Obmann/Obfrau berechtigt, auch in Angelegenheiten, die in den Wirkungsbereich der Vollversammlung oder der Generalversammlung oder des Vorstands fallen, unter eigener Verantwortung selbständig Anordnungen zu treffen;
        im Innenverhältnis bedürfen diese jedoch der nachträglichen Genehmigung durch das zuständige Vereinsorgan.

    \item Der/die Obmann/Obfrau führt den Vorsitz in der Generalversammlung und im Vorstand.

    \item Der/die Schriftführer/in führt die Protokolle der Generalversammlung und des Vorstands.

    \item Der/die Kassier/in ist für die ordnungsgemäße Geldgebarung des Vereins verantwortlich.

    \item Im Fall der Verhinderung treten an die Stelle des/der Obmanns/Obfrau, des Schriftführers/der Schriftführerin oder des Kassiers/der Kassierin ihre Stellvertreter/innen.
\end{enumerate}

\statutensection{16 Rechnungsprüfer/in}

\begin{enumerate}[statutenenum]
    \item Zwei Rechnungsprüfer/innen werden von der Generalversammlung auf die Dauer von 1 Jahr gewählt.
        Wiederwahl ist möglich.
        Die Rechnungsprüfer/innen dürfen keinem Organ -- mit Ausnahme der Generalversammlung und der Vollversammlung -- angehören, dessen Tätigkeit Gegenstand der Prüfung ist.

    \item Den Rechnungsprüfern/Rechnungsprüferinnen obliegt die laufende Geschäftskontrolle sowie die Prüfung der Finanzgebarung des Vereins im Hinblick auf die Ordnungsmäßigkeit der Rechnungslegung und die statutengemäße Verwendung der Mittel.
        Der Vorstand hat den Rechnungsprüfern/Rechnungsprüferinnen die erforderlichen Unterlagen vorzulegen und die erforderlichen Auskünfte zu erteilen.
        Die Rechnungsprüfer/innen haben dem Vorstand über das Ergebnis der Prüfung zu berichten.

    \item Rechtsgeschäfte zwischen Rechnungsprüfern/Rechnungsprüferinnen und Verein bedürfen der Genehmigung durch die Generalversammlung.
        Im Übrigen gelten für die Rechnungsprüfer/innen die Bestimmungen des \S\ 13 Abs.\ 8 bis 10 sinngemäß.
\end{enumerate}

\statutensection{17 Schiedsgericht}

\begin{enumerate}[statutenenum]
    \item Zur Schlichtung von allen aus dem Vereinsverhältnis entstehenden Streitigkeiten ist das vereinsinterne Schiedsgericht berufen.
        Es ist eine „Schlichtungseinrichtung“ im Sinne des Vereinsgesetzes 2002 und kein Schiedsgericht nach den \S\S\ 577 ff ZPO.

    \item Das Schiedsgericht setzt sich aus drei ordentlichen Vereinsmitgliedern zusammen.
        Es wird derart gebildet, dass ein Streitteil dem Vorstand ein Mitglied als Schiedsrichter/in schriftlich namhaft macht.
        Über Aufforderung durch den Vorstand binnen sieben Tagen macht der andere Streitteil innerhalb von 14 Tagen seinerseits ein Mitglied des Schiedsgerichts namhaft.
        Nach Verständigung durch den Vorstand innerhalb von sieben Tagen wählen die namhaft gemachten Schiedsrichter/innen binnen weiterer 14 Tage ein drittes ordentliches Mitglied zum/zur Vorsitzenden des Schiedsgerichts.
        Bei Stimmengleichheit entscheidet unter den Vorgeschlagenen das Los.
        Die Mitglieder des Schiedsgerichts dürfen keinem Organ -- mit Ausnahme der Generalversammlung -- angehören, dessen Tätigkeit Gegenstand der Streitigkeit ist.

    \item Das Schiedsgericht fällt seine Entscheidung nach Gewährung beiderseitigen Gehörs bei Anwesenheit aller seiner Mitglieder mit einfacher Stimmenmehrheit.
        Es entscheidet nach bestem Wissen und Gewissen.
        Seine Entscheidungen sind vereinsintern endgültig.
\end{enumerate}

\statutensection{18 Freiwillige Auflösung des Vereins}

\begin{enumerate}[statutenenum]
    \item Die freiwillige Aufösung des Vereins kann nur in einer Generalversammlung und nur mit Zweidrittelmehrheit der abgegebenen gültigen Stimmen beschlossen werden.

    \item Diese Generalversammlung hat auch -- sofern Vereinsvermögen vorhanden ist -- über die Abwicklung zu beschließen.
        Insbesondere hat sie eine Abwicklerin oder einen Abwickler zu berufen und Beschluss darüber zu fassen, wem dieser das nach Abdeckung der Passiven verbleibende Vereinsvermögen zu übertragen hat.

    \item Bei Auflösung des Vereins oder bei Wegfall des bisherigen begünstigten Vereinszwecks ist das verbleibende Vereinsvermögen für gemeinnützige oder mildtätige Zwecke im Sinne der \S\S\ 34ff BAO zu verwenden.
\end{enumerate}

\end{otherlanguage}
